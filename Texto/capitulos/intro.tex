\chapter{Introdução}


A Realidade Aumentada (\textit{Augmented Reality})
é um ramo da Realidade Virtual 
que vem sendo estudado nos últimos anos. Essa tecnologia
mescla os últimos avanços no Processamento de Imagens Digitais,
Inteligência Artificial, sensores de GPS, reconhecimento e
técnicas de interação humano-computador. Ela é largamente utilizada
em entretenimento, pesquisas científicas e militares, Medicina, manufatura e manutenção
de máquinas e muitas outras áreas \cite{ARFeatureMaching, DevActuallyRegistration}.

A Realidade Aumentada consiste na sobreposição de objetos virtuais, gerados por
computação, a imagens do mundo real, geralmente capturadas por câmeras digitais. Esse é
um processo em tempo real, como a exibição do placar na televisão durante uma partida
de futebol. 

Com a ajuda de técnicas avançadas de Realidade Aumentada, como visão computacional,
reconhecimento de objetos e geolocalização, as informações acerca da realidade que envolve o usuário
tornam-se interativas e digitalmente manipuláveis. Esses dados podem ser sobrepostos
às imagens que o usuário visualiza por meio de algum dispositivo de saída, como a tela de um
\textit{smartphone} ou \textit{tablet} \cite{AReX}.


A primeira interface usando Realidade Aumentada foi criada por Ivan Edward Sutherland, na década de 1960. 
Apesar disso, a primeira conferência especialmente dedicada a esse tema foi realizada apenas 
em 1998, o IWAR 98\footnote{International Workshop on Augmented Reality} \cite{TrendsInAR}. 
Sutherland, com ajuda de seu estudante Bob Sproull, criou o que foi considerado o primeiro dispositivo
de Realidade Aumentada e Realidade Virtual. Esse instrumento consistia no que é chamado de \gls{HMD}, uma
espécie de capacete, onde há uma tela acoplada, por meio da qual o usuário visualiza imagens geradas
por um computador. Esse projeto foi chamado de \textit{The Sword of Damocles} \cite{TheUltimateDisplay}.


Dispositivos móveis vêm ganhando muito espaço nos últimos anos, como pode ser visto segundo a 
pesquisa da \textit{Strategy Analytics}
\footnote{\href{http://www.strategyanalytics.com}{http://www.strategyanalytics.com}}. A pesquisa revelou que,
em outubro de 2012, já existiam mais de 1 bilhão de dispositivos móveis e que em até 3 anos será ultrapassada 
a marca de 2 bilhões. 

Dispositivos como \textit{smartphones} e \textit{tablets} vêm ganhando cada vez mais funcionalidades, com 
melhorias de \textit{hardwares}. Esses equipamentos podem ser utilizados para tarefas diversas e uma das
principais é a geolocalização, uma vez que recursos de localização, como \gls{GPS} e bússola, equipam a 
maioria desses aparelhos.

As principais soluções de geolocalização atuais, como o Google Maps, utilizam mapas 2D. Esse tipo de 
disposição de dados pode ser muito bom para navegação em grandes territórios. Porém, para locais mais
específicos, deveria haver uma aplicação mais próxima à realidade do usuário, por meio da qual fosse 
possível visualizar não apenas um mapa 2D da área onde ele está, mas, também, exibir as construções 
à sua volta, explicitando na tela do dispositivo a direção de cada ponto de interesse, bem como a distância
até eles.

Este trabalho descreve uma solução para esse problema. O objetivo é implementar um aplicativo, baseado em 
Realidade Aumentada, para geolocalização em 
um campus universitário. A solução desenvolvida utiliza a Realidade Aumentada para orientar o
usuário no Centro Politécnico da Universidade Federal do Paraná. A partir das coordenadas geográficas de alguns 
pontos de interesse, como blocos de salas de aula, departamentos e secretarias de cursos e Restaurante Universitário, 
a aplicação cria duas interfaces principais: um mapa 2D, onde cada ponto é identificado por um marcador, como na maioria
dos aplicativos de geolocalização; a outra interface utiliza a Realidade Aumentada. Ela exibe as imagens capturadas pela
câmera do dispositivo, sobrepostas com as identificações dos pontos de interesse do usuário. Conforme o usuário move o
equipamento, esses marcadores se movem na tela, orientando o usuário para qual direção cada ponto está, além da distância
até eles. A aplicação também permite cadastrar locais de interesse próprios do usuário, não restringindo o uso apenas com os
pontos de interesse fornecidos pelo sistema.









