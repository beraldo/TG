\chapter{Conclusão}
\label{chapter:conclusao}


Nota-se que a Realidade Aumentada é um tema que vem ganhando muito espaço
nos últimos 20 anos, principalmente em pesquisa acerca 
de técnicas de \textit{Tracking} e \textit{Registration} \cite{TrendsInAR}.
Existem diversas aplicações para a Realidade Aumentada, principalmente para
aprendizado e reconhecimento de ambientes. Embora ela venha crescendo lentamente,
cada vez mais pesquisadores se
interessam por sua evolução.

Com o avanço das tecnologias móveis, dos \textit{smartphone} e \textit{tablets}, 
a área de Realidade Aumentada Móvel deve crescer ainda mais, abrindo espaço para
novas categorias de aplicações, que auxiliarão os usuários nas tarefas do
dia-a-dia, como o \textit{Google Glass}, citado na Seção \ref{sec:trab_relacionados}.


Entre 1998 e 2007, de 100 artigos de Realidade Aumentada avaliados por \cite{TrendsInAR},
um total de 63 estudavam as técnicas de \textit{Tracking} e \textit{Registration}. Essas
áreas merecem atenção especial, por se tratarem de técnicas essenciais para o desenvolvimento
da Realidade Aumentada.


\section{Conclusões Acerca da Aplicação Desenvolvida}

O \textit{iPhone-AR-Toolkit} é uma ferramenta ainda em desenvolvimento e de integração não muito 
fácil com outras aplicações. Porém, vem recebendo diversas contribuições e, em um futuro próximo,
deve se tornar uma biblioteca de fácil implementações em aplicações de Realidade Aumentada. 

Um dos pontos fracos dessa ferramenta é a falta de integração com camadas de dados. Todas as 
informações de latitude e longitude estavam presentes dentro do código-fonte. Foi necessário
criar uma camada de dados e modificar a ferramenta para integrar-se com ela.

Outra falha aparente é a detecção da posição exata do usuário. Essa falha está relacionada com a precisão
do \gls{GPS}. Não se pode afirmar ao certo se é um problema da ferramenta ou do dispositivo usado. São 
necessários mais testes, em diferentes plataformas e localizações, para chegar a essa conclusão.

De modo geral, o \textit{iPhone-AR-Toolkit} mostrou-se estável e confiável para a aplicação proposta.



\section{Outras Possíveis Aplicações}

Esse tipo de aplicação permite explorar diversas outras áreas de interesse.
É possível usar a mesma técnica descrita aqui para guiar usuários em ambientes
turísticos, como museus, onde são exibidas na tela direções para pontos de interesse,
além das informações relacionadas a eles.

Também é possível adaptar essa técnica em sistemas de \gls{GPS} utilizados em veículos,
permitindo, além da visão aérea do mapa, uma visão da mesma perspectiva do usuário, onde 
o dispositivo orienta o usuário para qual direção ele deve ir a fim de chegar em seu destino.

Uma possibilidade muito interessante é no ramo de aplicações assistivas. Em vez de exibir na tela
locais conhecidos e suas direções até eles, o programa poderia orientar usuários cegos, por meio de
áudio, a fim de guiá-los em direção de seus destinos. 
